\documentclass[11pt, a4paper]{article}
\usepackage[utf8]{inputenc}  % Sometimes different char encoding systems are used. This package makes sure they work well
\usepackage{amsmath}
\usepackage{graphicx}
\usepackage{subfigure}
%\graphicspath{{.}}     % Set figure file path, when figures are located in the current path, it can be omitted.
\usepackage{booktabs}   % Make three-line table, the three line command need this package. Simple \hline don't need this package.
\usepackage[a4paper, top=2.5cm, bottom=2.5cm, left=2.2cm, right=2.2cm]{geometry}

\begin{document}
\section*{\centerline{MAEG3050 INTRODUCTION TO CONTROL SYSTEMS}}
\subsection*{\Large\centerline{Reference Solution to Problem Set\#2}}

\textbf{1. a)} With $\omega_n=8 \;rad/sec$, compute the real part and imaginary part of $p$ when $\zeta<1$:
\begin{center}
\centering
\begin{tabular}{cccccc}
\toprule
$\zeta$ &0 &0.2 &0.5 &0.7 &0.9\\
\midrule
$-\zeta\omega_n$ &0 &-1.6 &-4 &-5.6 &-7.2\\
$\omega_n\sqrt{1-\zeta^2}$ &8 &7.84 &6.93 &5.71 &3.49\\
\bottomrule  %\hline
\end{tabular}
\end{center}
When $\zeta = 1.0,$ $p = p^* = -8$, when $\zeta = 2,$ $p = -16\pm 8\sqrt{3}$, when $\zeta = 5,$ $p = -40\pm 16\sqrt{6}.$\\
\textbf{1. b)} \emph{(It can't be better if $\zeta$ is marked near the corresponding point.)}\\
\begin{figure}[h]
\includegraphics[scale = 0.3]{HW2_1b}
\centering
\end{figure}
\\
\textbf{2. a)} The Laplace transform of the system is
\begin{equation*}
\begin{split}
Y(s) & = \frac{sy(0)+(\dot{y}(0)+\frac{b}{M}y(0))}{s^2+\frac{b}{M}s+\frac{k}{M}}-\frac{\frac{F_0}{M_s}}{s(s^2+\frac{b}{M}s+\frac{k}{M})}\\
     & = \frac{A}{s-s_1}+\frac{B}{s-s_2}+\frac{E}{s}+\frac{C}{s-s_1}+\frac{D}{s-s_2} \\
     & = \frac{E}{s}+\frac{C}{s-s_1}+\frac{D}{s-s_2} \;\; (y(0)=0,\dot{y}(0)=0)
\end{split}
\end{equation*}
where $s_1$ and $s_2$ are the roots of equation $s^2+\frac{b}{M}s+\frac{k}{M} = 0$. The corresponding inverse Laplace transform is given by
\begin{equation*}
y(t)=E + Ce^{s_1t} + De^{s_2t}
\end{equation*}
The rate of covergence is determined by the slower item between $e^{s_1t}$ and $e^{s_2t}$. Usually, we consider the time the function
converges into $5\%$ range. Let
\begin{equation*}
e^{st}|_{t=0.5}=0.05
\end{equation*}
we get $s=-5.99$, so $s_1$ and $s_2$ should at least be smaller than -5.99. For instance, take $s_1=-12,s_2=-13$, then
\begin{equation*}
(s+12)(s+13)=s^2+\frac{b}{M}s+\frac{k}{M}
\end{equation*}
and $\frac{b}{M}=25,\frac{k}{M}=156$.\\
\emph{(Other reasonable choices of $s_1,s_2$ are also OK. Pay attention $s<-5.99$ is not sufficient to satisfy the requirement, some trials are needed to set smaller values of $s_1,s_2$. In this problem, rough trials show that s should be smaller than $-9.$)}
\\
\textbf{2. b)}
\begin{equation*}
\begin{split}
Y(s) &= -\frac{\frac{F_0}{M_s}}{s(s^2+\frac{b}{M}s+\frac{k}{M})}\\
     &= -\frac{4}{s(s+12)(s+13)}\\
     &= -\frac{4}{156s}+\frac{4}{12}\cdot\frac{1}{s+12}-\frac{4}{13}\cdot\frac{1}{s+13}
\end{split}
\end{equation*}
Perform the inverse Laplace transform and times $-\frac{k}{M}$,
\begin{equation*}
-\frac{k}{M}y(t) = 4-52e^{-12t}+48e^{-13t}
\end{equation*}
The plot is shown below.
\begin{figure}[h]
\includegraphics[scale = 0.64]{HW2_2b}
\centering
\end{figure}
\\
%\emph{(Pay attention $s<-5.99$ is not sufficient to satisfy the requirement, some trials are needed to set smaller values of $s_1,s_2$.)}
\\
\\
\textbf{3.} The block diagram from $V_a(s)$ to $\theta(s)$ is
\begin{figure}[h]
\includegraphics[scale = 0.3]{HW2_3a}
\centering
\end{figure}\\
The transfer function is
\begin{equation*}
\begin{split}
\frac{\theta(s)}{V_a(s)} &= \frac{\frac{K_m}{R_a+L_as}\frac{1}{Js+b}}{1+\frac{K_m}{R_a+L_as}\frac{1}{Js+b}K_b}\cdot\frac{1}{s}\\
                   			&= \frac{K_m}{s[(L_as+R_a)(Js+b)+K_mK_b]}
\end{split}
\end{equation*}
The two equivalent bolck diagrams from $T_d(s)$ to $\theta(s)$ are\\
\begin{figure}[h]
\includegraphics[scale = 0.35]{HW2_3b1}
\centering
\end{figure}\\
\begin{figure}[h]
\includegraphics[scale = 0.35]{HW2_3b2}
\centering
\end{figure}\\
The transfer function is
\begin{equation*}
\begin{split}
\frac{\theta(s)}{T_d(s)} &= -\frac{\frac{1}{Js+b}}{1+\frac{1}{J_s+b}\frac{K_m}{R_a+L_as}K_b}\cdot\frac{1}{s}\\
                   			&= -\frac{L_as+R_a}{s[(L_as+R_a)(Js+b)+K_mK_b]}
\end{split}
\end{equation*}
The angular position $\theta(s)$ is given by the sum of the outputs generated by $V_a(s)$ and $T_d(s)$, i.e.
\begin{equation*}
\theta(s) = \frac{K_m}{s[(L_as+R_a)(Js+b)+K_mK_b]}V_a(s) - \frac{L_as+R_a}{s[(L_as+R_a)(Js+b)+K_mK_b]}T_d(s)
\end{equation*} 
\\
\\
\textbf{4. a)} $p_1 = -3+j6$, i.e. $\sigma = 3, \omega_d = 6.$ So $\omega_n = \sqrt{\sigma^2+\omega_d^2} = 3\sqrt{5},\zeta 
= \frac{\sigma}{\omega_n}=\frac{1}{\sqrt{5}}.$
\\
\textbf{4. b)} $p_3 = -4cos(45^{\circ})+j4sin(30^{\circ})=-2\sqrt{2}+j2$, i.e. $\sigma = 2\sqrt{2}, \omega_d = 2.$ So $\omega_n = \sqrt{\sigma^2+\omega_d^2} = 2\sqrt{3},\zeta = \frac{\sigma}{\omega_n}=\sqrt{\frac{2}{3}}.$
\\
\\
\textbf{5.} One possible simplification process is shown below. So the transfer function for the block diagram is 
\begin{equation*}
\frac{R(s)}{C(s)} = \frac{G_2G_5(G_3-G_1)}{1+G_5G_6+G_2G_4G_5G_6(G_3-G_1)}
\end{equation*}
\begin{figure}[h]
\centering
\includegraphics[scale = 0.32]{HW2_5_1}
\end{figure}\\
\begin{figure}[h]
\centering
\subfigure{\includegraphics[scale = 0.32]{HW2_5_2}}
\subfigure{\includegraphics[scale = 0.32]{HW2_5_3}}
\subfigure{\includegraphics[scale = 0.32]{HW2_5_4}}
\subfigure{\includegraphics[scale = 0.32]{HW2_5_5}}
\subfigure{\includegraphics[scale = 0.32]{HW2_5_6}}
\end{figure}
\\
\\
\textbf{6.} The Laplace transform of the input and output are 
\begin{equation*}
U(s) = L(4te^{-2t}) = \frac{4}{(s+2)^2} 
\end{equation*}
\begin{equation*}
\begin{split}
Y(s) &= L(-1 + t + e^{-2t} + te^{-2t})\\
     &= -\frac{1}{s} + \frac{1}{s^2} + \frac{1}{s+2} + \frac{1}{(s+2)^2}\\
     &= \frac{4}{s^2(s+2)^2}
\end{split}
\end{equation*}
So the transfer function of the system is 
\begin{equation*}
\frac{Y(s)}{U(s)} = \frac{1}{s^2}
\end{equation*}
\\
\textbf{Q7 Recommended Solution1: using the dominant pole approximation}\\
\\
\textbf{7. a)}
\begin{equation*}
G(s) = \frac{1000}{s^2 + 1002s + 2000} = \frac{1000}{(s+2)(s+1000)} = \frac{500}{499}(\frac{1}{s+2} - \frac{1}{s+1000})
\end{equation*}
The inverse Laplace transform of $G(s)$ is given by
\begin{equation*}
g(t) = \frac{500}{499}(e^{-2t} - e^{-1000t})
\end{equation*}
Since $-1000$ is much smaller then $-2$, the dynamics component $e^{-1000t}$ will decay much faster than $e^{-2t}$.
\\
\textbf{7. b)}
\begin{equation*}
\begin{split}
G(s) &= \frac{1000}{(s+2)(s+1000)}\\
     &= \frac{0.5}{(0.5s+1)(0.001s+1)}\\
     &\approx \frac{0.5}{0.5s+1}\\
     &= \frac{1}{s+2}
\end{split}
\end{equation*}
for $0.5\gg0.001$, thus
\begin{equation*}
G_R(s) = \frac{1}{s+2}
\end{equation*}
\textbf{7. c)} Use the full transfer function $G(s)$.
\begin{equation*}
\begin{split}
Y(s) &= U(s)G(s) \\
     &= \frac{500}{499}\frac{1}{s}(\frac{1}{s+2} - \frac{1}{s+1000})\\
     &= \frac{1}{2s} - \frac{250}{499}\frac{1}{s+2} + \frac{1}{998}\frac{1}{s+1000}
\end{split}
\end{equation*}
Then $y(t)$ is given by
\begin{equation*}
y(t) = \frac{1}{2} - \frac{250}{499}e^{-2t} + \frac{1}{998}e^{-1000t}
\end{equation*}
Use the reduced transfer function $G_R(s)$. 
\begin{equation*}
\begin{split}
Y_R(s) &= U(s)G_R(s) \\
       &= \frac{1}{s(s+2)}\\
       &= \frac{1}{2}(\frac{1}{s}-\frac{1}{s+2})
\end{split}
\end{equation*}
Then $y(t)$ is given by
\begin{equation*}
y_R(t) = \frac{1}{2} - \frac{1}{2}e^{-2t}
\end{equation*}
\textbf{7. d)} Define the error/difference between the full output and reduced output as
\begin{equation*}
e = y(t) - y_R(t) = -\frac{1}{998}e^{-2t} + \frac{1}{998}e^{-1000t}
\end{equation*}
As $t$ grows, $e$ converges to 0 fast.\\
\\
\textbf{Q7 Solution2: using partial fraction expansion result}\\
\\
\textbf{7. a)}
\begin{equation*}
G(s) = \frac{1000}{s^2 + 1002s + 2000} = \frac{1000}{(s+2)(s+1000)} = \frac{500}{499}(\frac{1}{s+2} - \frac{1}{s+1000})
\end{equation*}
The inverse Laplace transform of $G(s)$ is given by
\begin{equation*}
g(t) = \frac{500}{499}(e^{-2t} - e^{-1000t})
\end{equation*}
Since $-1000$ is much smaller then $-2$, the dynamics component $e^{-1000t}$ will decay much faster than $e^{-2t}$.
\\
\textbf{7. b)} $G(s)$ is separated as two items $\frac{500}{499}\frac{1}{s+2}-\frac{500}{499}\frac{1}{s+1000}.$ Neglect
the fast dynamics $\frac{500}{499}\frac{1}{s+1000}.$ Then
\begin{equation*}
G_R(s) = \frac{500}{499(s+2)}
\end{equation*}
\textbf{7. c)} Use the full transfer function $G(s)$.
\begin{equation*}
\begin{split}
Y(s) &= U(s)G(s) \\
     &= \frac{500}{499}\frac{1}{s}(\frac{1}{s+2} - \frac{1}{s+1000})\\
     &= \frac{1}{2s} - \frac{250}{499}\frac{1}{s+2} + \frac{1}{998}\frac{1}{s+1000}
\end{split}
\end{equation*}
Then $y(t)$ is given by
\begin{equation*}
y(t) = \frac{1}{2} - \frac{250}{499}e^{-2t} + \frac{1}{998}e^{-1000t}
\end{equation*}
Use the reduced transfer function $G_R(s)$. 
\begin{equation*}
\begin{split}
Y_R(s) &= U(s)G_R(s) \\
     &= \frac{500}{499}\frac{1}{s}\frac{1}{s+2} \\
     &= \frac{250}{499}\frac{1}{s} - \frac{250}{499}\frac{1}{s+2}
\end{split}
\end{equation*}
Then $y(t)$ is given by
\begin{equation*}
y_R(t) = \frac{250}{499} - \frac{250}{499}e^{-2t}
\end{equation*}
\textbf{7. d)} Define the error/difference between the full output and reduced output as
\begin{equation*}
e = y(t) - y_R(t) = -\frac{1}{998} + \frac{1}{998}e^{-1000t}
\end{equation*}
As $t$ grows, $e$ converges to $-\frac{1}{998} \approx -0.001$, which is a very small error, quite fast.
\end{document}
